Since at least 2000 a class of large bright Ly$\alpha$ have been observed, termed Ly$\alpha$ Blobs.
There has been a lot of work on simulating these objects, but previous attempts have been conducted with a subset of relevant physics or at low resolution.
We present a set of Ly$\alpha$ Monte-Carlo radiative transfer simulations run on high-resolution MassiveFIRE snapshots which reproduce observed Ly$\alpha$ Blobs in luminosity and extent.
In analyzing the escaping emission, we find that it is dominated by emission from collisionally exicted neutral hydrogen, but in the presence of an AGN emission from recombinations dominates due to the elevated ionization state.
Additionally, the Ly$\alpha$ escape is highly sightline-dependent, and even more so in the presence of an AGN.
We also find that in our model, the presence of an AGN is well-predicted by the Gini coefficient, and therefore it may be possible to detect the presence of AGN in real observations with only surface brightness measurements.

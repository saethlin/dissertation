\appendix
\chapter*{APPENDIX \\ PingPong Example}
\addcontentsline{toc}{chapter}{APPENDIX: PINGPONG EXAMPLE}
\chaptermark{Appendix}
\markboth{Appendix}{Appendix}
\setcounter{chapter}{1}

\label{appendix:pingpong}
We present an example C++ program which can be used to estimate communication overhead of our work-scheduling system, mentioned in \ref{sec:methods}.

\begin{lstlisting}
#include <chrono>
#include <iostream>
#include <mpi.h>
#include <vector>

using namespace std::chrono_literals;

int rank, n_ranks, n_threads;

struct Photon {
    double a, b, c, d, e, f, g, h, i, j;
    int rank;
    int id;
};

const int OUTGOING_TAG = 1;
const int PHOTON_TAG = 2;
const int SHUTDOWN_TAG = 3;

const int n_photons = 10'000'000;

void be_worker() {

#pragma omp parallel
    {
        Photon p;
        MPI_Status status;

        while (true) {
#pragma omp critical
            MPI_Recv(&p, sizeof(Photon), MPI_UNSIGNED_CHAR, 0,
                     MPI_ANY_TAG, MPI_COMM_WORLD, &status);
            if (status.MPI_TAG == SHUTDOWN_TAG) {
                break;
            }
#pragma omp critical
            MPI_Send(&p, sizeof(Photon), MPI_UNSIGNED_CHAR, 0,
                     PHOTON_TAG, MPI_COMM_WORLD);
        }
    }
}

int main(int argc, char **argv) {
    int provided;

    n_threads = atoi(std::getenv("SLURM_CPUS_PER_TASK"));

    MPI_Init_thread(&argc, &argv, MPI_THREAD_SERIALIZED, &provided);
    if (provided != MPI_THREAD_SERIALIZED) {
        std::cout << "MPI thread initialization failed\n";
        return 1;
    }

    MPI_Comm_rank(MPI_COMM_WORLD, &rank);
    MPI_Comm_size(MPI_COMM_WORLD, &n_ranks);

    if (rank == 0) {
        std::vector<Photon> outgoing_photons(n_ranks * n_threads);
        std::vector<MPI_Request> send_requests(n_ranks * n_threads);
        std::vector<Photon> processed_photons;
        processed_photons.reserve(n_ranks + 1);
        int sent_photons = 0;

        auto start = std::chrono::high_resolution_clock::now();

        // Send an initial request to each worker
        // TODO: Don't submit to rank 1 at the moment to save on
        // locking in the main loop
        int id = 0;
        for (int r = 1; r < n_ranks; r++) {
            for (int t = 0; t < n_threads; t++) {
                outgoing_photons[id].id = id;
                outgoing_photons[id].rank = r;
                MPI_Isend(&outgoing_photons[id], sizeof(Photon),
                          MPI_UNSIGNED_CHAR, r, OUTGOING_TAG,
                          MPI_COMM_WORLD, &send_requests[id]);
                id++;
                sent_photons++;
            }
        }

        auto now = std::chrono::high_resolution_clock::now();
        std::cout << "Set up in " << (now - start).count() / 1e9
                  << " seconds\n";

        // Main event loop to pick up photons and send back a request
        // if more are required
        Photon p;
        while (sent_photons < n_photons) {
            MPI_Recv(&p, sizeof(Photon), MPI_UNSIGNED_CHAR,
                     MPI_ANY_SOURCE, PHOTON_TAG, MPI_COMM_WORLD,
                     MPI_STATUS_IGNORE);
            processed_photons.push_back(p);

            MPI_Request_free(&send_requests[p.id]);
            MPI_Isend(&outgoing_photons[p.id], sizeof(Photon),
                      MPI_UNSIGNED_CHAR, p.rank, OUTGOING_TAG,
                      MPI_COMM_WORLD, &send_requests[p.id]);
            sent_photons++;
        }

        // Pick up the remaining photons and tell processes to shut
        // down
        for (int i = 0; i < id; i++) {
            MPI_Recv(&p, sizeof(Photon), MPI_UNSIGNED_CHAR,
                     MPI_ANY_SOURCE, PHOTON_TAG, MPI_COMM_WORLD,
                     MPI_STATUS_IGNORE);
            processed_photons.push_back(p);

            MPI_Request_free(&send_requests[p.id]);
            MPI_Isend(&outgoing_photons[p.id], sizeof(Photon),
                      MPI_UNSIGNED_CHAR, p.rank, SHUTDOWN_TAG,
                      MPI_COMM_WORLD, &send_requests[p.id]);
        }

        now = std::chrono::high_resolution_clock::now();
        std::cout << "Got responses in "
                  << (now - start).count() / 1e9 << " seconds\n";

        for (int i = 0; i < id; i++) {
            MPI_Wait(&send_requests[i], MPI_STATUS_IGNORE);
        }

    } else {
        be_worker();
    }

    MPI_Barrier(MPI_COMM_WORLD);
    MPI_Finalize();
}
\end{lstlisting}

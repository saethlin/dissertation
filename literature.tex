\chapter{Thoughts on other published LAB studies and directions for the sub-field}
\label{literature}

This project was my first interaction with both cosmological simulations and Monte-Carlo radiative transfer.
I had no experience whatsoever with the field when my advisor, Desika Narayanan pitched me the original project.
And it was pitched as an attempt to build a model that accounts for more phenomena, which would enable us to make a more detailed study of which of these phenomena contribute to the luminosity and the spatial extent that is so characteristic of LABs.
This was the lens through which I did my initial study of the literature; I was paying close attention to how many physical phenoena various papers were able to claim they had accounted for.

Over the course of this project, my perspective on the progress this work represents with respect to the rest of the literature has changed.
Whereas before I was focused on how many boxes we could check, I now find much more interesting \emph{how} these phenomena are manifested in the simulation.
I think it is a general trend over all work in simulation to replace sub-grid models with a first-principles model.
